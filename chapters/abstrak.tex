\clearpage
\phantomsection
\addcontentsline{toc}{section}{ABSTRAK}
\section*{ABSTRAK}

Transformasi digital pada sektor Usaha Mikro, Kecil, dan Menengah (UMKM) menjadi faktor krusial dalam meningkatkan efisiensi operasional dan daya saing pasar. UMKM Ice Tea menghadapi tantangan dalam pengelolaan data transaksi dan pemantauan inventaris yang masih dilakukan secara konvensional, sehingga berisiko menimbulkan ketidakakuratan data. Penelitian ini bertujuan untuk merancang dan mengimplementasikan sistem informasi penjualan berbasis web dengan fitur utama automasi manajemen stok dan penyediaan dashboard analitik. Pengembangan sistem ini dibangun menggunakan Framework Laravel yang berbasis PHP dengan penerapan arsitektur MVC (Model-View-Controller) untuk pemrosesan logika pada sisi backend. Pendekatan ini dipilih untuk meningkatkan keamanan, keteraturan kode, dan skalabilitas sistem. Sisi frontend menggunakan mesin templat Blade yang terintegrasi dengan HTML, CSS, dan JavaScript untuk antarmuka pengguna. Seluruh data operasional diintegrasikan menggunakan sistem manajemen basis data MySQL dengan fitur Eloquent ORM guna memastikan manipulasi data yang efisien dan aman. Hasil penelitian menunjukkan bahwa integrasi fitur pemesanan online dengan pembaruan stok secara otomatis mampu meminimalisir kesalahan manusia dalam pencatatan barang. Selain itu, dashboard analitik yang disediakan membantu pemilik usaha dalam memantau tren penjualan secara real-time untuk mendukung pengambilan keputusan strategis. Laporan ini diharapkan dapat menjadi rujukan praktis dalam digitalisasi tata kelola bisnis bagi pelaku UMKM.

\textbf{Kata Kunci}: UMKM, Website, Laravel, MySQL, MVC, Manajemen Stok Otomatis, Dashboard Analitik.
